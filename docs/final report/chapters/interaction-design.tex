\chapter{Interaction Design} %MAX 2 PAGES
In this chapter we will describe the interaction design test we have done with the user. First we will describe the persona of a typical user of the tool and next we will tell how the test. Furthermore we will draw the conclusions from the test.

\section{Methods}
We put the user behind the computer and asked him to perform tasks. We first wanted him to read in files and then perform analyses. If the user can not figure out how to do something, we gave some hints to help him along the way. The more complicated analyses were pre-written so that the user only has to modify the existing analysis to make it work. We did this because the language we use has a high learning curve and we did not have a lot of time at the test to make the user learn the complete language.

\section{Expectation}
In this section we will describe what we expected how the test would go. We expected that the test could be a bit too difficult. This is because the the language we use is not that easy to understand at the first time using it. We provided the users with some example scripts of the language to make this process easier.
We expected that the graphical user interface would be easier to understand.

\section{General Persona}
In this section we will describe John Doe who is an abstraction of our typical user.\\
John is an analyst and he uses analysis tools on a regular basis. His analyses are performed on data that is collected during research. He has a specific goal in mind and he wants to get to an answer to a specific question about the data. John also has minimal experience in programming, but is familiar with scripts from other analysis tools and is eager to learn a new scripting language.

\section{Evaluations}
% bruikbaarheids evaluatie

%Describe our method of evaluation here (friday evaluations, high fidelity prototype etc)
% conclusies uit de testen

% zelfs met voorbeelden alsnog moeilijk evaluaties toe te passen