\chapter{Engineering Reflection} %MAX 1 PAGE

Introduce the chapter, explain what will be descibed. 2 PAGES

\section{GitHub Pulls}
It was not allowed to push changes directly to the master. When a person wanted to merge a change into the master he had to open a pull request. The pull request had to be approved by at least two team members before it could be merged with the master. The code for the pull request should contain comments and javadoc. Furthermore it should be tested and the code should have a good quality. And at last, the tools we use to evaluate the quality of the code, must say that the code was good. We look very critical to pull request. So often the one who had requested the pull requested had to improve some parts of his code. This led to good code quality. However most comments were related to code style and not directly to the implementation of the code.
The size of the pull requests varies a lot. In the beginning of the project they were very large. But since they were large, reviewing was difficult. And since they were large it was not possible to do the review quick, so people postponed the reviews. Therefore we tried to create smaller pull requests. This result of this was that pull requests were handled quicker and that we could review them better. However there were still some very large pull requests.
The time that a pull request was open varies a lot. Small reviews were most of the time handled within an hour. However large pull request stayed sometimes open for a very long time. For large pull request it could take sometimes days before it was reviewed and corrected again. So therefore we tried to keep the pull request small. The reviewing of the code, led to good code quality. 
\section{Code Quality}
Talk about the way we maintained quality in the code

\section{SIG Feedback}
Talk about how we made changes after SIG feedback 