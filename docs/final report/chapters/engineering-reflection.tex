\chapter{Engineering Reflection} %MAX 1 PAGE

In this section we'll describe how the engineering process went. We discuss how we've used scrum to develop the product. Furthermore we'll discuss the GitHub pull requests and how we ensured that our code was of good quality. Finally we will discuss the feedback we got from SIG.

\section{Scrum}
We used the scrum work flow to develop the product. Friday we created a plan for the coming week. Every week-day we started with a short dailymeeting. In that meeting we discussed what we'd done, what we were going to do and which problems we'd encountered. On Friday our sprint ended. We created a sprintreflection and we gave a demo to the client. Next we started on new sprint. 
At the start of the project we underestimated the time needed for tasks. Therefore a lot of task weren't done during a sprint. Later in the project we got better at estimating the time required and therefore there were less incompleted tasks.

We liked that scrum was focused on features. But we also noticed that because the focus on features, there is no room for task that can not be translated into a feature. Though those task could be of importance.

\section{GitHub Pulls}
It wasn't allowed to push changes directly to the master. When a person wanted to merge into the master he had to open a pull request. The pull request had to be approved by at least two team members before it could be merged with the master. The code for the pull request should be of good quality (see section \ref{codequality}).  We looked very critical at pull request. This meant that the person who'd performed the pull request had to improve some parts of his code. However most comments were related to code style and not directly to the implementation of the code. But still the reviews led to improvement of the code quality. 

The size of the pull requests varies a lot. In the beginning of the project they were very large. This made them hard to review and caused them to stay open for a long time. Therefore we tried to create smaller pull requests. This result of this was that pull requests were handled quicker and that we could review them better. However there were still some very large pull requests.

Reviewing the code resulted in good code quality. Therefore we will try use pull based development in future projects. We'll remember to keep the pull requests small. 

\section{Code Quality}\label{codequality}
During the project we tried to create code of good quality. The code had to meet a few requirements before it was considered to be good quality. First of all the code had to follow the conventions. Furthermore the code should be clear and self-explaining. If the code was not self-explaining it needed comments. The code had to be tested and had to contain JavaDoc. And at last it should work efficiently and it had to be easy to make changes.
 
To achieve this we used some tools. The first tool we used was checkstyle. Checkstyle reported the situations where we did not follow the code conventions. For instance if there were unused imports, checkstyle would report that. 
Another tool we used was PMD. PMD detects a variety of code smells. For instance it detects unused variables and methods, duplicated code and a lot of other things.
Furthermore we used FindBugs. FindBugs is capable of detecting possible errors.

Before a person opened a pull request he had to fix the issues reported by the tools. Furthermore the code should've been tested. Then other team members would look into the code and make some suggestions to improve the code quality.
In general this worked pretty well, and led to good code quality. However the tools were not always run, therefore it happened sometimes that some errors detected by the tools ended up in the master branch. 

\section{SIG Feedback}
In week 7 we got feedback from SIG. In general our code quality was good and above average. On duplication and on component balance we scored below average. By the time we got the feedback, we already refactored a part of the code that caused a lot of duplication. Since the teacher of the context project did not found the component balance a big issue. We decided that it was not worth to invest time to improve this. However we have tried to restructure our packages a bit better. 

Since our overall feedback was good, we concluded that the way we ensured our code quality worked and there was no need to change that process.
Near the end of the project we focused less on quality and more on features. This has done some harm to the code quality.