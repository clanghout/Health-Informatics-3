\chapter{Results of Interaction Design Tests}\label{ch:results}
\begin{itemize}
\item \textbf{Mark} is a student in the same year as us. He is also a participant of the context project but he works on the Programming Life context.\\\\
The first thing that is noticed is that in the help button in the menu bar the about button does not function. Mark is linked to the external manual for our program. Then he got the exercise to change an existing XML file using our wizard. It took some time for him to find the load function. It is actually located in the menu bar in the wizard. We asked to copy the settings of one file to another and then Mark wondered why the "copy template" and "paste template" buttons are next to the path of the file.\\
Then an analysis was loaded into the internal editor. The analysis was kind of hard to understand but he recognised the form of the definitions.\\
Next we went to the visualization part. We started by letting him create a bar chart. It turned out to be not so clear where the columns selected stand for. While creating a box plot he discovered a bug, the error message when selecting a non number column does not disappear when a correct column is selected and the pop-up has to be closed and reopened to create the box plot. After this nothing worth mentioning happened and we finished the test.
\item \textbf{Robin} is also a student in the same study year. He is on the same context as us, so he knows well what the purpose of our program is.\\\\
Robin also got the task to change an already existing XML file. He quickly found the load button in the menu bar, but it was not clear that files can be selected to change its contents. When adding files he wondered why the buttons for the files are "browse" and "remove" instead of "add" and "remove". When he edited the path of the file, it was not clear that the apply button had to be pressed to confirm this changes. Because he started with loading an XML file, he wondered why he had to specify a location when the "create XML" button was pressed, because he expected to automatically override the existing file. When copy pasting templates in the wizard, he sometimes had to press the copy button twice before the template was actually copied. He wondered if he could get feedback when copying the template but than stated that normal the normal computer copy also gives no visible feedback.\\
At the analysis a pre made analysis was loaded. He had quite a hard time understanding the language. Robin wondered where the vertical lines in our pipe system were for. To help him we showed the language specification, where our pipe system is explained. After that a simple constraint analysis was made by him. While saving the analysis Robin noted that the file selectors always go to the home folder where as their program saves the previous selected location for user friendliness. After some scripts were loaded and slightly adjusted, codes were set on the demo table. \\
With the codes set we gave the exercise to create a state-transition-matrix. When in the pop-up for creating the matrix he asked the reason why he had to select a column so we pointed to the label above the selector which he missed. He told us that he liked the option to select the codes to use for the matrix. After the matrix was created and saved, one last remark was made by Robin. He asked why the default value in the selectors for columns for the visualizations are not already columns. After this the test was over. 
\end{itemize}