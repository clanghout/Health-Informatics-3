\documentclass[a4paper]{article}

\usepackage[english]{babel}
\usepackage[utf8]{inputenc}
\usepackage{amsmath}
\usepackage{graphicx}
\usepackage{listings}
\usepackage[colorinlistoftodos]{todonotes}
\title{Product Planning}

\usepackage{enumitem}
\usepackage{authblk}
\setlist{nolistsep}
\author[1]{Boudewijn van Groos}
\author[2]{Chris Langhout}
\author[3]{Jens Langerak}
\author[4]{Paul van Wijk}
\author[5]{Louis Gosschalk}

\affil[1]{bvangroos \\
	4229843}
\affil[2]{clanghout \\
	4281705}
\affil[3]{jlangerak \\
	4317327}
\affil[4]{pjvanwijk \\
	4285034}
\affil[5]{lgosschalk \\
	4214528}


\date{\today}

\begin{document}
\maketitle
\begin{abstract}
	This document describes the product planning. It describes the requirements by using the MoSCoW method and by making use of user stories. Furthermore it gives a planning of the releases and it describes definition of done for the backlog items, sprints and releases.
\end{abstract}
\newpage
\tableofcontents
\newpage

\section{Introduction}
This document discusses the product planning. In the first section it describes the requirements according to the MoSCoW method. Section \ref{subsec:User-stories} contains the user stories for some of the requirements. Furthermore this document contains the planning for the product. In section \ref{subsec: roadmap} there is a high-level overview of the roadmap and section \ref{subsec: release-plan} gives a detailed planning for each week. Section \ref{sec: Definition-of-Done} explains what the definition of done is for a backlog item, for a sprint and for a release. 
\section{Product}
This section describes the requirements of the product. It also contains a high-level roadmap.
\subsection{High-level product backlog}
In this section we describe the product backlog according the MoSCoW method. Therefor we divide the features into four group. The features are divided based on their priority. Section 3 describes for some features a user story.
\subsubsection{Must Haves}
These features are essential for the product. Without these features the product is not usable.
\begin{itemize}
  \item Language in which the user can describe different analysis
  \item Executing an analysis that is defined in a file.
  \item Load data in the program based on a description file.
  \item Indicating the data connections between the different datafiles
  \item Indicating the meaning of the various data inputs
  \item Load data from different sources using the one data description file
  \item The 8 C's for data analysis
  \begin{itemize}
    \item Chunk analysis
    \item Comments
    \item Codes 
    \item Connections
    \item Comparisons
    \item Constraints
    \item Conversions
    \item Computations
  \end{itemize}
  \item Specifying the output and output format
  \item Visualizations from the data analyzed
  \begin{itemize}
  	\item Frequency bars
    \item Line graph
  \end{itemize}
  \item Manual for the analysis description language
\end{itemize}

\subsubsection{Should Haves}
These features are very useful. However without these features the product is still usable.
\begin{itemize}
	\item Visualizations
    \begin{itemize}
    	\item Box plot
        \item Stem leave
        \item State transition matrix
        \item Lag analysis
    \end{itemize}
	\item Exporting the visualizations to images
    \item Implement some example analysis in our analysis description language
\end{itemize}

\subsubsection{Could Haves}
These features will only be done when there is enough time.
 \begin{itemize}
  \item Visualizations
  \begin{itemize}
	  \item Histogram
      \item Markov chain graph transition diagram dingen
  \end{itemize}
  \item Editor for inputting the analysis description 
  \item Mass input for batch processing 
  \item Preview of the output from the analysis
\end{itemize}

\subsubsection{Would Haves}
These features will not be implemented during this project. If this project is followed up by another project, these feature might be interesting.
\begin{itemize}
	\item Easy to use GUI for specifying the analysis 
\end{itemize}

\subsection{Roadmap}
\label{subsec: roadmap}
This section will describe the planning for the product. In this roadmap we will plan the mayor releases. For a detailed overview of the tasks for each week see section \ref{subsec: release-plan}. The numbers of the week correspond to the week of quarter 4. A week ends on Friday.\\
Week 5 and 7 does not have a specific release goal, in those weeks we should work on the goals for the next week. The goals of week 6 and 8 are too large to achieve in one week. Thus week 5 and 7 should be used to achieve those goals.\\\\
\begin{tabular}[H]{ l p{10cm} }
Week & goals \\
\hline
1 & Setup project \\
2 & Minimal design of the user interface and basic architecture of the product \\
3 & Product vision and the program must be able to read and write data\\
4 & Product planning and it must be possible to perform the most important data analyses. \\
6 & The user can perform all the types of data analyses and the most important visualizations can be shown \\
8 & The program is able to show all planned visualizations \\
9 & Final product \\
10 & Final report and presentation
\end{tabular}

\section{Product backlog}
\subsection{User stories}
\subsubsection{Must}
As a user, I want to be able to specify the analysis I want to perform, so that I get relevant results from the program. 
\\\\
As a user, I want to be able to load the data in the program that should be analyzed, so that the analysis are performed over relevant data. 
\\\\
As a user, I want to be able to specify the relation between different data file, so that I can uses multiple files in an analysis.
\\\\
As a user, I want to be able to specify which columns are dates or point of time and in what format the columns is, so that I can perform analyses based in the date or time.
\\\\
As a user, I want to be able to specify which columns are numeric, so that I can perform analyses based on numeric values
\\\\
As a user, I want to be able to specify the name of the data columns, so that I can use those names in the analysis.
\\\\
As a user, I want to be able to create a file that specifies which files should be read for the data that is used for an analysis, so that I can load multiple files at once.
\\\\
As a user, I want to be able to perform analysis based on the eight c of sequential data analysis, so that I can perform sequential data analysis on the data.
\\\\
As a user, I want to be able write the output of an analysis to a file, so that I can uses the result for further analyses.
\\\\
As a user, I want to be able to specify which columns should be written to a file, so that I can discard not relevant data.
\\\\
As a user I want to be able to view the data as Frequency bars, so that I can see how often a event happens.
\\\\
As a user, I want to be able to view the data as a Line Graph, so that I can see how the behavior of the analyzed person changed over time.
\\\\
As a user, I want to have a manual that specifies the languages that is used for analyses.

\subsubsection{should}
As a user, I want to be able to export a visualization as an image, so that I can use the image in reports.

\subsubsection{Could}
As a user, I want to be able to perform one analysis over multiple datasets, so that I can perform the same analysis over all statsensors at once.

As a user, I want to be able to edit an analysis in the program, so that I don't have to reload the analysis when I change it.

As a user, I want to preview the output of an analysis, so that I immediately can see if the result is what I expected.

As a user, I want a GUI for specifying the analyses, so that I don't have to program the analysis myself.

\subsection{Initial release plan}
\label{subsec: release-plan}
This section will describe the planning for the product. The release plan is based on sprints of one week and on the roadmap described in section \ref{subsec: roadmap}. The numbers of the week correspond to the week of quarter 4. A new iteration starts on every Friday. For each week we will list which features the product should have and which additional task must be done. Furthermore for each feature we will list the priority. This is the same priority as in section \ref{subsec: MoSCoW}.
\subsubsection{Week 4.1}
\begin{itemize}
	\item Setup the software that is used during the project
	\item Obtain the requirements
\end{itemize}

\subsubsection{Week 4.2}
\begin{itemize}
\item A basic architecture for the product
\item A design for the user interface
\item A draft version of the product vision
\end{itemize}
\subsubsection{Week 4.3}
\begin{itemize}
	\item A minimal user interface according the design of week 4.2
	\item The final version of the product vision
	\item A draft version of the product planning
	\item The user must be able to specify in a data description file how a file should be read by the program (Must Have)
	\item The user must be able to specify which data must be written to a file (Must Have)
	\item The user must be able to perform constraint analyses (Must Have)
\end{itemize}
\subsubsection{Week 4.4}
\begin{itemize}
	\item The final version of the product planning
	\item The user must be able to perform chunking analyses (Must Have)
	\item The user must be able to perform connections analyses (Must Have)
	\item The user must be able to perform computation analyses (Must Have)
\end{itemize}
\subsubsection{Week 4.5}
\begin{itemize}
	\item The user must be able to perform codes analyses (Must Have)
	\item The user must be able to perform comparisons analyses (Must Have)
	\item It must be possible to show the data as frequency bars (Must Have)
	\item It must be possible to show the data as a line graph (Must Have)
\end{itemize}
\subsubsection{Week 4.6}
\begin{itemize}
	\item The user must be able to perform comments analyses (Must Have)
	\item The user must be able to perform conversions analyses (Must Have)
	\item It must be possible to show the data as a box plot (Should Have)
	\item It must be possible to show the data as a Stem-and-Leaf plot (Should Have)
	\item Input for SIG
\end{itemize}
\subsubsection{Week 4.7}
\begin{itemize}
	\item It must be possible to show the data as a state transition matrix (Should Have)
	\item Show the data with Lag analysis (Should Have)
	\item The user must be able to export the visualizations as an image (Should Have)
	\item It must be possible to show the data as a Histogram (Could Have)
\end{itemize}
\subsubsection{Week 4.8}
\begin{itemize}
	\item It must be possible to show the data as a Markov chain (Could Have)
	\item Specify multiple files that all will be analyzed individually (Could Have)
	\item Implement certain analyses functions in our language
\end{itemize}
\subsubsection{Week 4.9}
	This is the last week where it is possible to work on the code. No new features are planned for this week. In this way we will be able to handle some delay during the process. Furthermore this week is used to repair the last bugs. Therefore there is a feature freeze on Wednesday June 17. 
\begin{itemize}
	\item Final input for SIG
	\item Draft version of the final report
\end{itemize}
\subsubsection{Week 4.10}
\begin{itemize}
	\item Final report
	\item Product presentation
\end{itemize}
\section{Definition of Done}
\label{sec: Definition-of-Done}
In this section we will discuss when a task is considered as done. In general a task is done when there is nothing left to do for that task. We will discuss the definition of done for backlog items, sprints and releases.

\subsection{Backlog Item}
\label{subsec: Done-Item}
A backlog item is done if it is implemented as described and it follows the description of the user stories. Furthermore the code should have been tested with unit tests. All the other features should still work and all the tests should pass. The code must be reviewed by at least two persons who have not worked on that specific item. The code should be clear and when needed, it should contain comments. Furthermore the code should follow the languages conventions and it should have clear names for the variables. When the item meets all these requirements, than it is considered done.

\subsection{Sprint}
Each sprint should have a sprint plan and a sprint reflection. Any deliverable that has a due in or at the end of the sprint should have been made and handed in. Furthermore  if needed, relevant documents, such as the architecture design, should have been updated. Critical bugs that are discovered during the sprint must be fixed. If it is not possible to fix them during the sprint, than they have to be solved in the next sprint. Finally all the task of the sprint should be completed as described in the previous section. If it is not possible to complete a certain task in a sprint, than the sprint reflection should explain why it is was not possible to finish the task.

\subsection{Release}
Each sprint ends with a new version of the product. Sections 2.2 and 3.5 provide an overview of the planned features for each release. Based on that, each sprint will add some new features to the product.
A release is only allowed to contain features that are considered done, see section \ref{subsec: Done-Item}. Therefore all the features in a release are tested and the code should be proper. Additionally we must test whether the features work correctly together. Furthermore a release may not have any critical bug. Finally for each release a demo has to be prepared and demonstrated. 



\end{document}