\subsection{High-level product backlog}
\label{subsec: MoSCoW}
In this section we describe the product backlog according to the MoSCoW method. Therefore the features are divided into four groups. The features are divided based on their priority. Section \ref{subsec: release-plan} describes for some features a user story. For each feature or group of features we will also explain why it has that priority.
\subsubsection{Must Haves}
These features are essential for the product. Without these features the product is not usable. \\
The user wants to be able to perform different kind of analysis, so it must be possible to define the analysis that is to be perform. If this is not possible the program is useless. Also the user must know how the language works, therefore a manual is essential.
\begin{itemize}
  \item Language in which the user can describe different analyses
  \item Executing an analysis that is defined in a file
  \item Load data in the program based on a description file
  \item Manual for the analysis description language
\end{itemize}
The data that is collected during the research, is stored in multiple files. Therefore it must be possible to use multiple files in an analysis.
\begin{itemize}
  \item Indicating the data connections between the different datafiles
  \item Load data from different sources using the one data description file
\end{itemize}
There are different types of data. For example, there are strings, numbers and dates. Not all operations can be done on every types. So it must be possible to distinguish them.
\begin{itemize}
  \item Indicating the meaning of the various data inputs
\end{itemize}
It must be possible to do analysis. Therefore the program must implement some data analysis techniques.
\begin{itemize}
  \item The 8 C's for exploratory data analysis
  \begin{itemize}
    \item Chunk analysis
    \item Comments
    \item Codes 
    \item Connections
    \item Comparisons
    \item Constraints
    \item Conversions
    \item Computations
  \end{itemize}
    \end{itemize}
The user wants to use the output of the program in other programs. So this program must be able to output the result in such way that other programs can use it.
  \begin{itemize}
  \item Specifying the output and output format
\end{itemize}
The user wants that the program can visualize the data in different ways. However not all visualization were equally important to the user. The visualizations are prioritized in the way the user suggested.
\begin{itemize}
  \item Visualizations from the analyzed data
  \begin{itemize}
  	\item Frequency bars
    \item Line graph
  \end{itemize}
\end{itemize}

\subsubsection{Should Haves}
These features are very useful. However without these features the product is still usable.
\begin{itemize}
	\item Visualizations
    \begin{itemize}
    	\item Box plot
        \item Stem leaf
        \item State transition matrix
        \item Lag analysis
    \end{itemize}
 \end{itemize}
   When the user wants to use the generated visualizations in a document. He needs some way to create a image of the visualization.
 \begin{itemize}
	\item Exporting the visualizations to images
\end{itemize}
Some examples in the analysis language will help the user understand the language better. However there is already a manual, so this is a should have and not a must have.
\begin{itemize}
    \item Implement some example analyses in our analysis description language
\end{itemize}

\subsubsection{Could Haves}
These features will only be done when there is enough time.
 \begin{itemize}
  \item Visualizations
  \begin{itemize}
	  \item Histogram
      \item Markov chain
      \item Transition diagram
  \end{itemize}
\end{itemize}
When the user want to edit a analysis file, it will be very inconvenient if the user has to edit the file outside the program and then reload it in the program. It would be nice if the user is able to edit the file in the program. Furthermore the user wants to analyze multiple files. It is convenient if it is possible to perform the analysis over all files at once. However both mentioned features will need a lot of time to be implemented. Therefore we consider this as a could have.
\begin{itemize}
  \item Editor for inputting an analysis description 
  \item Mass input for batch processing 
  \item Preview of the output from the analyses
\end{itemize}

\subsubsection{Would Haves}
These features will not be implemented during this project. If this project is followed up by another project, these features might be interesting.\\
A scripting language for the analysis might be difficult to understand for a user. Therefore the user would profit from a simple GUI to specify the analysis. However this will cost a lot of time to implement and is therefore out of the scope of this project. 
\begin{itemize}
	\item Easy to use GUI for specifying the analyses 
\end{itemize}
