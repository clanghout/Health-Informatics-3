\subsection{High-level product backlog}
In this section we describe the product backlog according the MoSCoW method. Therefor we divide the features into four group. The features are divided based on their priority. Section 3 describes for some features a user story.
\subsubsection{Must Haves}
These features are essential for the product. Without these features the product is not usable.
\begin{itemize}
  \item Language in which the user can describe different analysis
  \item Executing an analysis that is defined in a file.
  \item Load data in the program based on a description file.
  \item Indicating the data connections between the different datafiles
  \item Indicating the meaning of the various data inputs
  \item Load data from different sources using the one data description file
  \item The 8 C's for data analysis
  \begin{itemize}
    \item Chunk analysis
    \item Comments
    \item Codes 
    \item Connections
    \item Comparisons
    \item Constraints
    \item Conversions
    \item Computations
  \end{itemize}
  \item Specifying the output and output format
  \item Visualizations from the data analyzed \todo{Dit is niet noodzakelijk voor het product, dus should?}
  \begin{itemize}
  	\item Frequency bars
    \item Line graph
  \end{itemize}
  \item Manual for the analysis description language
\end{itemize}

\subsubsection{Should Haves}
These features are very useful. However without these features the product is still usable.
\begin{itemize}
	\item Visualizations
    \begin{itemize}
    	\item Box plot
        \item Stem leave
        \item State transition matrix
        \item Lag analysis
    \end{itemize}
	\item Exporting the visualizations to images \todo{should or could?}
    \item Preimplement \todo{is this a correct english word?} codes in our analysis description language
\end{itemize}

\subsubsection{Could Haves}
These features will only be done when there is enough time.
 \begin{itemize}
  \item Visualizations
  \begin{itemize}
	  \item Histogram
      \item Markov chain graph transition diagram dingen
  \end{itemize}
  \item Editor for inputting the analysis description \todo{deze doen we niet, won't?}
  \item Mass input for batch processing \todo{should?}
  \item Preview of the output from the analysis
\end{itemize}

\subsubsection{Would Haves}
These features will not be implemented during this project. If this project is followed up by another project, these feature might be interesting.
\begin{itemize}
	\item Easy to use GUI for specifying the analysis \todo{womwisselen met de editor?}
\end{itemize}
