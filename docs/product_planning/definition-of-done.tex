\section{Definition of Done}
\label{sec: Definition-of-Done}
In this section we will discuss when a task is considered as done. In general a task is done when there is nothing left to do for that task. We will discuss the definition of done for backlog items, sprints and releases.

\subsection{Backlog Item}
\label{subsec: Done-Item}
A backlog item is done if it is implemented as described and it follows the description of the user stories. Furthermore the code should have been tested with unit tests. All the other features should still work and all the tests should pass. The code must be reviewed by at least two persons who have not worked on that specific item. The code should be clear and when needed, it should contain comments. Furthermore the code should follow the languages conventions and it should have clear names for the variables. When the item meets all these requirements, than it is considered done.

\subsection{Sprint}
Each sprint should have a sprint plan and a sprint reflection. Any deliverable that has a due in or at the end of the sprint should have been made and handed in. Furthermore  if needed, relevant documents, such as the architecture design, should have been updated. Critical bugs that are discovered during the sprint must be fixed. If it is not possible to fix them during the sprint, than they have to be solved in the next sprint. Finally all the task of the sprint should be completed as described in the previous section. If it is not possible to complete a certain task in a sprint, than the sprint reflection should explain why it is was not possible to finish the task.

\subsection{Release}
Each sprint ends with a new version of the product. Sections 2.2 and 3.5 provide an overview of the planned features for each release. Based on that, each sprint will add some new features to the product.
A release is only allowed to contain features that are considered done, see section \ref{subsec: Done-Item}. Therefore all the features in a release are tested and the code should be proper. Additionally we must test whether the features work correctly together. Furthermore a release may not have any critical bug. Finally for each release a demo has to be prepared and demonstrated. 