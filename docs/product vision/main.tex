\documentclass[a4paper]{article}

\usepackage[english]{babel}
\usepackage[utf8]{inputenc}
\usepackage{amsmath}
\usepackage{graphicx}
\usepackage[colorinlistoftodos]{todonotes}
\usepackage{authblk}

\title{Product Vision}

\author[1]{Boudewijn van Groos}
\author[2]{Chris Langhout}
\author[3]{Jens Langerak}
\author[4]{Paul van Wijk}
\author[5]{Louis Gosschalk}

\affil[1]{bvangroos \\
4229843}
\affil[2]{clanghout \\
4281705}
\affil[3]{jlangerak \\
4317327}
\affil[4]{pvanwijk \\
4285034}
\affil[5]{lgosschalk \\
4214528}

\date{\today}

\renewcommand*\contentsname{Summary}

\begin{document}
\maketitle
\tableofcontents
\newpage

\section{Target Customer}
The target customer of our software is a group of skilled researching analysts of the LUMC. The researchers are currently working on a system for renal transplant patients to keep track of their transplanted organ's values and thus keeping track of its health. These researchers want to be able to analyze their patient's behavior. The analysis of the patient's behavior stretches from a case of manipulation of data to the extent of a patient's 'obedience' to a system's advice.

\section{Customer's Needs}
The product will address the need of the analysts to generate usable values such as means and averages from abstract data and create visualizations of this generated data. The customer needs to be able to 
\begin{itemize}
\item load data from different sorts of sources consisting of varying structures 
\item create their own analysis 
\item execute their analysis on the loaded data
\item create visualizations of the analyzed data
\item output the analyzed data to a data format usable in additional software (e.g. SPSS).
\end{itemize}

\section{Crucial Attributes}
To satisfy the user's needs, it is crucial that the software will be able to recognize and process data of which the structure varies. It is also necessary that the software will offer a scripting language which enables the user to create their analysis which then can be used on the loaded data. This language should be easy to learn. Thus it has to be documented properly. The user will be able to see visualizations of their analysis which must be clear and meaningful. After the loading of data, creation of analysis, execution of analysis and visualization of the analysis results, the software must also be able to export the analysis results into a file so it can be used in additional analysis software. The following attributes are crucial.
\begin{itemize}
\item Input module accepting specified data sources
\item Simplistic analysis language
\item Clear documentation of the language
\item Visualize the analyzed data
\item Export module with possibility to specify output syntax
\end{itemize}

\section{Unique Selling Points}
To know what makes our software unique, it is necessary to know which alternatives are on the market. When searching for data analysis tools, the names SPSS, SAS, R and Excel are the most prominent. The tool proving to be most relevant seems to be SPSS due to frequency of appearance. After an extensive search for a general definition of SPSS, the following description was found: 
\begin{quotation}
"SPSS is a Windows based program that can be used to perform data entry and analysis
and to create tables and graphs. SPSS has scores of statistical and mathematical functions,
scores statistical procedures, and a very flexible data handling capability. Some of the
functionalities of SPSS are: Descriptive Statistics, Reliability tests, Correlation, T-tests,
Regression and curve estimation." \cite{spssdescription}
\end{quotation}
\par 
This description immediately shows some points on which we can improve with our software. For example; SPSS is windows based. Our software is written in Java and thus it can be used on any operating system. Another point of improvement is the list of functionalities that are given. Our target customer doesn't need these functionalities, so it would make the software needlessly complicated.
\par
So the key to what makes our software unique is that it's personalized and therefor extremely usable. It is specifically created for analysts to process the data from renal monitoring devices, hospital records and the renal monitoring website. The provided language for creating analysis can be kept concise and simple because the purpose is so specific, the graphical user interface can be kept clean because it desires not too many extended features, which any other data analysis tool already has. 
\par 
The company itself does not yet have software that compares with our software, but there are other groups working on the same assignment. This means we will have to outshine them in some way. Since the other groups have received the exact same assignment and the same description for the software's exact functionality requirements, we will need to create a software of high quality and usability to 'win'. We plan to do this by creating an easy and effective GUI and implementing as much automation in routine actions as possible. This will be achieved by using User Centered Design principles\cite{nielsen}.


\section{Time-frame}
There's a time-frame of 10 weeks for the entire project. We aim to finish the product in week 9 so we will have the time to debug in week 10. Apart from the debugging, week 10 is very chaotic due to presentations and meetings so there will probably not be a lot of time in that week to develop a lot. We also aim to have a working version continuously throughout the process of development. There is no budget, the team works free of charge and even pays tuition to be able to do this project.

\begin{thebibliography}{9}
\bibitem{spssdescription} 
Khairy H. Abdelkareem and N. AL-Mekhlafy (2014)
\textit{Roof lateral displacement for gravity load designed rc frames subjected to earthquakes} Assiut, Assiut University.

\bibitem{nielsen}
J. Nielsen (1993) 
\textit{Usability Engineering}
Boston, Academic Press.
\end{thebibliography}

\end{document}