\subsection{Reading and writing tables}
The from process has been in various examples. It simply reads the table given in the parameter and outputs it into the next process.

The is process does the opposite of the from, it takes an input and saves it to the model.

So renaming a table:
$from(admire2)|is(testTable)$

\subsection{Sorting tables}
Sorting a table is done through the sort process. 
For example: $from(test1)|sort(test1.value, "ASC")|is(sorted)$

Is you'd wanted it descending just replace "ASC" with "DESC".

\subsection{Set operations}
We support both the union and difference set operations.

So the following will add one table to another and then remove all the rows to return to the original: $union(test1, test2)|difference(test1, test2)$