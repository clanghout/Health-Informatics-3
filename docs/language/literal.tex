Since all languages require some hardcoding to make everyone's live easier, we also support it. You have several types of literals.

\subsection{String literals}
Example: $"Hello"$ \\
String literals are denoted by $""$s. Unfortunately that means we don't support the use of $"$ in string literals. We encourage any future developers to work on supporting this.

\subsection{Integer literals}
Example: $5$ \\
Integer literals are quite simple. They are any number not containing a dot.

\subsection{Float literals}
Example: $5.0$ \\
Float literals are also doable. They are any number containing a dot.

\subsection{Date literals}
Examples: $\#1995-01-17\#$, $\#13:33\#$, $\#1995-01-23\ 23:35:54\#$

Dates are more tricky. We only support one date format which is year-month-day. This is to prevent ambiguity between the day-month-year and month-day-year formats. The time format is hour:minutes:seconds or hour:minutes. In the latter case the seconds are set to 0.

Dates are always enclosed between hashtags.

Combined the two will result in a datetime.

\subsection{Period literals}
Examples: $\#5\ DAYS\#$, $\#0\ MOTNHS\#$, $\#5\ YEARS\#$

Periods are a number of days, months or years. This is most useful when adding or subtracting a period of time from a date.

Periods are also enclosed between hashtags and the supported units are DAYS, MONTHS and YEARS.