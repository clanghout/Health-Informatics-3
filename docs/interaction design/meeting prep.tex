\documentclass[a4paper]{article}

\usepackage[english]{babel}
\usepackage[utf8]{inputenc}
%styling of author list
\usepackage{authblk}


\title{Interaction Design Meeting}

\author[1]{Louis Gosschalk}
\author[2]{Boudewijn van Groos}
\author[3]{Jens Langerak}
\author[4]{Chris Langhout}
\author[5]{Paul van Wijk}

\affil[1]{lgosschalk \\
4214528} 
\affil[2]{bvangroos \\
4229843}
\affil[3]{jlangerak \\
4317327}
\affil[4]{clanghout \\
4281705}
\affil[5]{pvanwijk \\
4285034}
\affil[ ]{Health Informatics Group C}

\date{\today}

\renewcommand*\contentsname{Summary}

\begin{document}
\maketitle

\section{Relevant}
the Final Report will also
contain a section describing the HCI module that was realized for the user interaction with
the developed solution. This section will reveal what the students learned in the Interaction
Design course and will be evaluated by the corresponding lecturer. The grade for Interaction
Design will be assigned based on the content of this section (see Section 8 for assessment
process and criteria). $\to$ Dit mag dus ongeveer 2 a4'tjes omvatten; zie: \ref{sec:Referentie}:\ref{itm:ite}

De volgende punten zijn interessant voor het $ID$ deel van het final report: 
\begin{enumerate}
\item User Centered Design: \begin{itemize}
	\item vroeg in het proces naar gebruiker en context gekeken
	\item evaluatie met wenxin gehad
	\item iteratief ontwerp
	\item mono-disciplinair team
	\end{itemize}
\item reden voor user involvement is voornamelijk vanuit pragmatisch en commitment rationale
\item contextual inquiry  \begin{itemize}
	\item requirements zijn voor ons opgesteld door Willem-Paul
	\item gesprekken met wenxin ivm haar werk
	\end{itemize}
\item cultural probe : niet toegepast
\item persona : beschrijving van Wenxin Wang
\item We maken gebruik van scenario's in de vorm van \begin{itemize}
	\item conceptuele scenario's (uitwerkvoorbeelden)
	\item use cases (sprint plans)
	\item concrete scenario's (demo met wenxin)
	\item gebruikersverhalen (ongebruikt)
	\end{itemize}
\item high-fidelity prototyping 
\item empirical evaluations (demo with wenxin) through debriefing
\item predictive evaluation (prototype)
\item geen rekening gehouden met perceptuele aspecten en motoriek (irrelevant voor ons project)
\item geheugen is relevant bij bijv. het werken met de taal. Hiermee houden we rekening dmv de language manual
\item hoe hebben wij in het programma rekening gehouden met menselijk falen?
\item houden geen rekening met ontwerp voor kinderen.
\item houden ook geen rekening met kleurenblindheid omdat de doelgroep zeer specifiek is.
\item we houden geen rekening met emotionele toestanden omdat het een professioneel programma is
\item We houden wel rekening met flow, door zo veel mogelijk handelingen te automatiseren
\item antropomorfisme of collaboratie, social awareness en persuasive technology n.v.t.
\end{enumerate}

\section{Beschrijving Final Report} \label{sec:Referentie}
This deliverable is the main document about the developed, implemented, and validated
software product. It will present the main functionalities of the product and discuss to which
extent they satisfy the needs of the user. For this purpose, an evaluation of the functionalities
performed using a well-justified method needs to be presented, as well as a failure analysis --
where the product does not perform as needed. Furthermore, the Final Report will also
contain a section describing the HCI module that was realized for the user interaction with
the developed solution. This section will reveal what the students learned in the Interaction
Design course and will be evaluated by the corresponding lecturer. The grade for Interaction
Design will be assigned based on the content of this section (see Section 8 for assessment
process and criteria). Finally, an outlook will be given regarding the possible improvements in
the future and the strategy to achieve these improvements.
Note that this report should not repeat the material from Product Vision, but should
complement it by providing results as response to expectations and strategy described in the
Product Vision document. 
The TOC of the Final Report will be as follows:
\begin{enumerate}
\item Introduction, including a brief problem description and end-user's requirements - 1 page
\item Overview of the developed and implemented software product - 1 page
\item Reflection on the product and process from a software engineering perspective - 1 page
\item Description of the developed functionalities - 2 pages
\item Special section on interaction design (development of the HCI module) - 2 pages \label{itm:ite}
\item Evaluation of the functional modules and the product in its entirety, including the - 2 pages
failure analysis
\item Outlook - 1 page
\end{enumerate}

\end{document}