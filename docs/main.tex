\documentclass[a4paper]{article}

\usepackage[english]{babel}
\usepackage[utf8]{inputenc}
\usepackage{amsmath}
\usepackage{graphicx}
\usepackage{listings}
\usepackage[colorinlistoftodos]{todonotes}
\title{MoSCoW}

\author{Health Informatics 3}

\date{\today}

\begin{document}
\maketitle


\section{Must Haves}
\begin{itemize}
  \item Language for description the analysis
  \item Reading in the analysis description file
  \item Reading in the data description file
  \item Indicating the data connections between the different datafiles
  \item Indicating the meaning of the various data inputs
  \item Reading in data from different sources using the data description file
  \item The 8 C's for data analysis
  \begin{itemize}
    \item Chunk analysis
    \item Comments
    \item Codes 
    \item Connections
    \item Comparisons
    \item Constraints
    \item Conversions
    \item Computations
  \end{itemize}
  \item Specifying the output and output format
  \item Visualizations from the data analyzed
  \begin{itemize}
  	\item Frequency bars
    \item Line graph
  \end{itemize}
  \item Manual for the analysis description language
\end{itemize}

\section{Should Haves}
\begin{itemize}
	\item Visualizations
    \begin{itemize}
    	\item Box plot
        \item Stem leave
        \item State transistion matrix
        \item Lag analysis
    \end{itemize}
	\item Exporting the visualizations to images
    \item Preimplemented codes in our analysis description language
\end{itemize}

\section{Could Haves}
\begin{itemize}
  \item Visualizations
  \begin{itemize}
	  \item Histogram
      \item Markov chain graph transition diagram dingen
  \end{itemize}
  \item Editor for inputting the analysis description
  \item Mass input for batch processing
  \item Preview of the output from the analysis
\end{itemize}

\section{Would Haves}
\begin{itemize}
	\item Handy GUI for specifying the analysis
\end{itemize}



\end{document}